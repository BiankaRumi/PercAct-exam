% Options for packages loaded elsewhere
\PassOptionsToPackage{unicode}{hyperref}
\PassOptionsToPackage{hyphens}{url}
%
\documentclass[
]{article}
\usepackage{lmodern}
\usepackage{amssymb,amsmath}
\usepackage{ifxetex,ifluatex}
\ifnum 0\ifxetex 1\fi\ifluatex 1\fi=0 % if pdftex
  \usepackage[T1]{fontenc}
  \usepackage[utf8]{inputenc}
  \usepackage{textcomp} % provide euro and other symbols
\else % if luatex or xetex
  \usepackage{unicode-math}
  \defaultfontfeatures{Scale=MatchLowercase}
  \defaultfontfeatures[\rmfamily]{Ligatures=TeX,Scale=1}
\fi
% Use upquote if available, for straight quotes in verbatim environments
\IfFileExists{upquote.sty}{\usepackage{upquote}}{}
\IfFileExists{microtype.sty}{% use microtype if available
  \usepackage[]{microtype}
  \UseMicrotypeSet[protrusion]{basicmath} % disable protrusion for tt fonts
}{}
\makeatletter
\@ifundefined{KOMAClassName}{% if non-KOMA class
  \IfFileExists{parskip.sty}{%
    \usepackage{parskip}
  }{% else
    \setlength{\parindent}{0pt}
    \setlength{\parskip}{6pt plus 2pt minus 1pt}}
}{% if KOMA class
  \KOMAoptions{parskip=half}}
\makeatother
\usepackage{xcolor}
\IfFileExists{xurl.sty}{\usepackage{xurl}}{} % add URL line breaks if available
\IfFileExists{bookmark.sty}{\usepackage{bookmark}}{\usepackage{hyperref}}
\hypersetup{
  pdftitle={PercAct exam},
  hidelinks,
  pdfcreator={LaTeX via pandoc}}
\urlstyle{same} % disable monospaced font for URLs
\usepackage[margin=1in]{geometry}
\usepackage{color}
\usepackage{fancyvrb}
\newcommand{\VerbBar}{|}
\newcommand{\VERB}{\Verb[commandchars=\\\{\}]}
\DefineVerbatimEnvironment{Highlighting}{Verbatim}{commandchars=\\\{\}}
% Add ',fontsize=\small' for more characters per line
\usepackage{framed}
\definecolor{shadecolor}{RGB}{248,248,248}
\newenvironment{Shaded}{\begin{snugshade}}{\end{snugshade}}
\newcommand{\AlertTok}[1]{\textcolor[rgb]{0.94,0.16,0.16}{#1}}
\newcommand{\AnnotationTok}[1]{\textcolor[rgb]{0.56,0.35,0.01}{\textbf{\textit{#1}}}}
\newcommand{\AttributeTok}[1]{\textcolor[rgb]{0.77,0.63,0.00}{#1}}
\newcommand{\BaseNTok}[1]{\textcolor[rgb]{0.00,0.00,0.81}{#1}}
\newcommand{\BuiltInTok}[1]{#1}
\newcommand{\CharTok}[1]{\textcolor[rgb]{0.31,0.60,0.02}{#1}}
\newcommand{\CommentTok}[1]{\textcolor[rgb]{0.56,0.35,0.01}{\textit{#1}}}
\newcommand{\CommentVarTok}[1]{\textcolor[rgb]{0.56,0.35,0.01}{\textbf{\textit{#1}}}}
\newcommand{\ConstantTok}[1]{\textcolor[rgb]{0.00,0.00,0.00}{#1}}
\newcommand{\ControlFlowTok}[1]{\textcolor[rgb]{0.13,0.29,0.53}{\textbf{#1}}}
\newcommand{\DataTypeTok}[1]{\textcolor[rgb]{0.13,0.29,0.53}{#1}}
\newcommand{\DecValTok}[1]{\textcolor[rgb]{0.00,0.00,0.81}{#1}}
\newcommand{\DocumentationTok}[1]{\textcolor[rgb]{0.56,0.35,0.01}{\textbf{\textit{#1}}}}
\newcommand{\ErrorTok}[1]{\textcolor[rgb]{0.64,0.00,0.00}{\textbf{#1}}}
\newcommand{\ExtensionTok}[1]{#1}
\newcommand{\FloatTok}[1]{\textcolor[rgb]{0.00,0.00,0.81}{#1}}
\newcommand{\FunctionTok}[1]{\textcolor[rgb]{0.00,0.00,0.00}{#1}}
\newcommand{\ImportTok}[1]{#1}
\newcommand{\InformationTok}[1]{\textcolor[rgb]{0.56,0.35,0.01}{\textbf{\textit{#1}}}}
\newcommand{\KeywordTok}[1]{\textcolor[rgb]{0.13,0.29,0.53}{\textbf{#1}}}
\newcommand{\NormalTok}[1]{#1}
\newcommand{\OperatorTok}[1]{\textcolor[rgb]{0.81,0.36,0.00}{\textbf{#1}}}
\newcommand{\OtherTok}[1]{\textcolor[rgb]{0.56,0.35,0.01}{#1}}
\newcommand{\PreprocessorTok}[1]{\textcolor[rgb]{0.56,0.35,0.01}{\textit{#1}}}
\newcommand{\RegionMarkerTok}[1]{#1}
\newcommand{\SpecialCharTok}[1]{\textcolor[rgb]{0.00,0.00,0.00}{#1}}
\newcommand{\SpecialStringTok}[1]{\textcolor[rgb]{0.31,0.60,0.02}{#1}}
\newcommand{\StringTok}[1]{\textcolor[rgb]{0.31,0.60,0.02}{#1}}
\newcommand{\VariableTok}[1]{\textcolor[rgb]{0.00,0.00,0.00}{#1}}
\newcommand{\VerbatimStringTok}[1]{\textcolor[rgb]{0.31,0.60,0.02}{#1}}
\newcommand{\WarningTok}[1]{\textcolor[rgb]{0.56,0.35,0.01}{\textbf{\textit{#1}}}}
\usepackage{graphicx,grffile}
\makeatletter
\def\maxwidth{\ifdim\Gin@nat@width>\linewidth\linewidth\else\Gin@nat@width\fi}
\def\maxheight{\ifdim\Gin@nat@height>\textheight\textheight\else\Gin@nat@height\fi}
\makeatother
% Scale images if necessary, so that they will not overflow the page
% margins by default, and it is still possible to overwrite the defaults
% using explicit options in \includegraphics[width, height, ...]{}
\setkeys{Gin}{width=\maxwidth,height=\maxheight,keepaspectratio}
% Set default figure placement to htbp
\makeatletter
\def\fps@figure{htbp}
\makeatother
\setlength{\emergencystretch}{3em} % prevent overfull lines
\providecommand{\tightlist}{%
  \setlength{\itemsep}{0pt}\setlength{\parskip}{0pt}}
\setcounter{secnumdepth}{-\maxdimen} % remove section numbering

\title{PercAct exam}
\author{}
\date{\vspace{-2.5em}}

\begin{document}
\maketitle

\begin{Shaded}
\begin{Highlighting}[]
\KeywordTok{.libPaths}\NormalTok{( }\KeywordTok{c}\NormalTok{(}\StringTok{"C:/Users/biank/Documents/Skole/CogSci/R_packages"}\NormalTok{, }\KeywordTok{.libPaths}\NormalTok{() ) )}
\KeywordTok{.libPaths}\NormalTok{()}
\end{Highlighting}
\end{Shaded}

\begin{verbatim}
## [1] "C:/Users/biank/Documents/Skole/CogSci/R_packages"    
## [2] "C:/Users/biank/OneDrive/Dokumenter/R/win-library/3.6"
## [3] "C:/Program Files/R/R-3.6.2/library"
\end{verbatim}

\begin{Shaded}
\begin{Highlighting}[]
\NormalTok{pacman}\OperatorTok{::}\KeywordTok{p_load}\NormalTok{(tidyverse)}
\NormalTok{pacman}\OperatorTok{::}\KeywordTok{p_load}\NormalTok{(mFilter)}
\end{Highlighting}
\end{Shaded}

\hypertarget{r-markdown}{%
\subsection{R Markdown}\label{r-markdown}}

3.2. The following data represent the right-hand movements of the person
in 3.1 while moving towards the bottle to grasp it. We are viewing the
person from the side so only y (back-front axis) and z (up-down axis)
are given. (Note: The measurement units are meaningless in this
example.) Download the two demo trajectories: 105\_1.txt and 105\_2.txt.
Then do the following steps in R. Include your commented code and the
resulting figures in your overall exam document, i.e.~do not just send a
link to your GitHub etc. {[}10\%{]} o Load the two trajectories and
rename the columns in each to `time', `y', `z'.

\begin{Shaded}
\begin{Highlighting}[]
\CommentTok{#we read the data}

\NormalTok{t1<-}\KeywordTok{read.delim}\NormalTok{(}\StringTok{"105_1.txt"}\NormalTok{,}\DataTypeTok{header=}\NormalTok{F,}\DataTypeTok{sep=}\StringTok{","}\NormalTok{)}

\NormalTok{t2<-}\KeywordTok{read.delim}\NormalTok{(}\StringTok{"105_2.txt"}\NormalTok{,}\DataTypeTok{header=}\NormalTok{F,}\DataTypeTok{sep=}\StringTok{","}\NormalTok{)}

\KeywordTok{names}\NormalTok{(t1)[}\KeywordTok{names}\NormalTok{(t1) }\OperatorTok{==}\StringTok{ "V1"}\NormalTok{] <-}\StringTok{ "time"}
\KeywordTok{names}\NormalTok{(t1)[}\KeywordTok{names}\NormalTok{(t1) }\OperatorTok{==}\StringTok{ "V2"}\NormalTok{] <-}\StringTok{ "y"}
\KeywordTok{names}\NormalTok{(t1)[}\KeywordTok{names}\NormalTok{(t1) }\OperatorTok{==}\StringTok{ "V3"}\NormalTok{] <-}\StringTok{ "z"}


\KeywordTok{names}\NormalTok{(t2)[}\KeywordTok{names}\NormalTok{(t2) }\OperatorTok{==}\StringTok{ "V1"}\NormalTok{] <-}\StringTok{ "time"}
\KeywordTok{names}\NormalTok{(t2)[}\KeywordTok{names}\NormalTok{(t2) }\OperatorTok{==}\StringTok{ "V2"}\NormalTok{] <-}\StringTok{ "y"}
\KeywordTok{names}\NormalTok{(t2)[}\KeywordTok{names}\NormalTok{(t2) }\OperatorTok{==}\StringTok{ "V3"}\NormalTok{] <-}\StringTok{ "z"}
\end{Highlighting}
\end{Shaded}

o Plot y versus z for the two curves on top of each other within the
same graph. Add meaningful labels.

\begin{Shaded}
\begin{Highlighting}[]
\KeywordTok{ggplot}\NormalTok{()}\OperatorTok{+}\KeywordTok{geom_point}\NormalTok{(}\DataTypeTok{data=}\NormalTok{t1, }\KeywordTok{aes}\NormalTok{(}\DataTypeTok{x=}\NormalTok{y,}\DataTypeTok{y=}\NormalTok{z, }\DataTypeTok{color=}\StringTok{"Trajectory 1"}\NormalTok{), }\DataTypeTok{color=}\StringTok{'green'}\NormalTok{)}\OperatorTok{+}\KeywordTok{geom_point}\NormalTok{(}\DataTypeTok{data=}\NormalTok{t2, }\KeywordTok{aes}\NormalTok{(}\DataTypeTok{x=}\NormalTok{y,}\DataTypeTok{y=}\NormalTok{z, }\DataTypeTok{color=}\StringTok{"Trajectory 2"}\NormalTok{), }\DataTypeTok{color=}\StringTok{'red'}\NormalTok{)}\OperatorTok{+}\KeywordTok{labs}\NormalTok{(}\DataTypeTok{x=}\StringTok{"Y trajectory"}\NormalTok{, }\DataTypeTok{y=}\StringTok{"Z trajectory"}\NormalTok{)}
\end{Highlighting}
\end{Shaded}

\includegraphics{percact_files/figure-latex/unnamed-chunk-3-1.pdf}

o For the first trajectory only: o Calculate z velocity and plot it
against time in a new graph.

\begin{Shaded}
\begin{Highlighting}[]
\CommentTok{#Calculating velocity}

\CommentTok{# Calculating z velocity. Velocity = delta y / delta t}
\NormalTok{t1}\OperatorTok{$}\NormalTok{z_velocity<-(t1}\OperatorTok{$}\NormalTok{z}\OperatorTok{-}\KeywordTok{lag}\NormalTok{(t1}\OperatorTok{$}\NormalTok{z))}\OperatorTok{/}\NormalTok{(t1}\OperatorTok{$}\NormalTok{time}\OperatorTok{-}\KeywordTok{lag}\NormalTok{(t1}\OperatorTok{$}\NormalTok{time))}

\CommentTok{#Plotting the z velocity against time}
\KeywordTok{ggplot}\NormalTok{()}\OperatorTok{+}\KeywordTok{geom_point}\NormalTok{(}\DataTypeTok{data=}\NormalTok{t1, }\KeywordTok{aes}\NormalTok{(}\DataTypeTok{x=}\NormalTok{time,}\DataTypeTok{y=}\NormalTok{z_velocity), }\DataTypeTok{color=}\StringTok{'green'}\NormalTok{)}
\end{Highlighting}
\end{Shaded}

\begin{verbatim}
## Warning: Removed 1 rows containing missing values (geom_point).
\end{verbatim}

\includegraphics{percact_files/figure-latex/unnamed-chunk-4-1.pdf}

o Apply a Butterworth filter to the z velocity. Choose reasonable
parameters yourself.

\begin{Shaded}
\begin{Highlighting}[]
\CommentTok{# Load library with butterworth filter}
\KeywordTok{library}\NormalTok{(signal)}
\end{Highlighting}
\end{Shaded}

\begin{verbatim}
## Warning: package 'signal' was built under R version 3.6.3
\end{verbatim}

\begin{verbatim}
## 
## Attaching package: 'signal'
\end{verbatim}

\begin{verbatim}
## The following object is masked from 'package:dplyr':
## 
##     filter
\end{verbatim}

\begin{verbatim}
## The following objects are masked from 'package:stats':
## 
##     filter, poly
\end{verbatim}

\begin{Shaded}
\begin{Highlighting}[]
\CommentTok{# Filter settings}
\NormalTok{filter_cutoff <-}\StringTok{ }\FloatTok{.1}
\NormalTok{filter_order <-}\StringTok{  }\DecValTok{2}
\NormalTok{bf <-}\StringTok{ }\KeywordTok{butter}\NormalTok{(filter_order, filter_cutoff, }\DataTypeTok{type =}\StringTok{'low'}\NormalTok{) }

\CommentTok{# Applying filter to z velocity}
\NormalTok{t1}\OperatorTok{$}\NormalTok{z_velocity_f<-}\StringTok{"filtfilt"}\NormalTok{(bf, t1}\OperatorTok{$}\NormalTok{z_velocity)}

\CommentTok{#we try to plot the butterworth-filtered z_velocity}
\KeywordTok{ggplot}\NormalTok{()}\OperatorTok{+}\KeywordTok{geom_point}\NormalTok{(}\DataTypeTok{data=}\NormalTok{t1, }\KeywordTok{aes}\NormalTok{(}\DataTypeTok{x=}\NormalTok{time,}\DataTypeTok{y=}\NormalTok{z_velocity_f), }\DataTypeTok{color=}\StringTok{'yellow'}\NormalTok{)}
\end{Highlighting}
\end{Shaded}

\includegraphics{percact_files/figure-latex/unnamed-chunk-5-1.pdf}

o Plot the filtered curve against time on top of the unfiltered one in
the previous graph.

\begin{Shaded}
\begin{Highlighting}[]
\CommentTok{#Yellow is filtered, black is not filtered}
\KeywordTok{ggplot}\NormalTok{()}\OperatorTok{+}\KeywordTok{geom_point}\NormalTok{(}\DataTypeTok{data=}\NormalTok{t1, }\KeywordTok{aes}\NormalTok{(}\DataTypeTok{x=}\NormalTok{time,}\DataTypeTok{y=}\NormalTok{z_velocity_f), }\DataTypeTok{color=}\StringTok{'yellow'}\NormalTok{)}\OperatorTok{+}\KeywordTok{geom_point}\NormalTok{(}\DataTypeTok{data=}\NormalTok{t1, }\KeywordTok{aes}\NormalTok{(}\DataTypeTok{x=}\NormalTok{time,}\DataTypeTok{y=}\NormalTok{z_velocity), }\DataTypeTok{color=}\StringTok{'black'}\NormalTok{)}\OperatorTok{+}\KeywordTok{labs}\NormalTok{(}\DataTypeTok{x=}\StringTok{"Time"}\NormalTok{,}\DataTypeTok{y=}\StringTok{"Z Velocity"}\NormalTok{, }\DataTypeTok{color=}\StringTok{"Filtered or not"}\NormalTok{)}
\end{Highlighting}
\end{Shaded}

\begin{verbatim}
## Warning: Removed 1 rows containing missing values (geom_point).
\end{verbatim}

\includegraphics{percact_files/figure-latex/unnamed-chunk-6-1.pdf}

\end{document}
